%----------------------------------------------------------------------------------------
%	PAQUETES GENERALES
%----------------------------------------------------------------------------------------
\usepackage[spanish]{babel} 			% Configuracion para palabras en Español (Le quite es-noshorthands)
\usepackage{microtype} 					% Mejor tipografia
\usepackage{amsmath,amsfonts,amsthm} 	% Paquetes para matematicas
\usepackage{graphicx} 					% Requerido para agregar imagenes
\usepackage{float}						% Requerido para el posicionamiento flotante
\usepackage{tcolorbox}					% Permite hacer recuadros para definiciones
\usepackage{multicol}					% Multicolumnas
\usepackage[hidelinks]{hyperref}		% Para el manejo de los hipervinculos
\usepackage{color}						% Permite colocar colores a los textos
\usepackage{circuitikz}					% Permite crear circuitos electricos

%----------------------------------------------------------------------------------------
%	TIPOGRAFIA
%----------------------------------------------------------------------------------------
% OPCION 1 -> Usa la fuente instalada en el sistema
\usepackage{fontspec}						% Configura la fuente. 
\setmainfont{Roboto}						% Para usar Arial, instalar ttf-mscorefonts-installer

% Bloque Inline de Codigo
\definecolor{light-gray}{gray}{0.95}
\newcommand{\code}[1]{\colorbox{light-gray}{\lstinline|#1|}}



%----------------------------------------------------------------------------------------
%	MARGINS AND SPACING
%----------------------------------------------------------------------------------------
\usepackage{geometry}
\geometry{
	top=1cm,
	bottom=1cm,
	left=1cm,
	right=1cm,
	includehead,
	includefoot,
	%showframe, % Descomentar para visualizar los limites
}

% \setlength{\columnsep}{5mm} % Separacion entre columnas
\usepackage{titlesec}						% Permite modificar los espacios de los titulos
\titlespacing{\section}{0pt}{20pt}{5pt}		% Izquierda del titulo, Antes y despues del mismo


\usepackage{afterpage}

\def\blankpage{%
	\clearpage%
	\addtocounter{page}{-1}%
	\null%
	\clearpage}


%----------------------------------------------------------------------------------------
%	ENCABEZADO Y PIE DE PAGINA
%----------------------------------------------------------------------------------------
\usepackage{fancyhdr}	% Permite modificar los Encabezados y Pie de Pagina
\usepackage{lastpage} 	% Used to determine the number of pages in the document (for "Page X of Total")
\fancypagestyle{plain}{
	\fancyhf{}
	\renewcommand{\headrulewidth}{1pt}
	\renewcommand{\footrulewidth}{1pt}
	\fancyfoot[OR, EL]{Pagina \thepage/\pageref*{LastPage}}
	\fancyfoot[OL, ER]{Christian Yoel Herrera}
	\fancyhead[OR, EL]{\textsc{\fecha}}
	\fancyhead[OL, ER]{\tituloHeader}
}

\newcommand{\blankPage}{
	\fancyhf{}
	\renewcommand{\headrulewidth}{0pt}
	\renewcommand{\footrulewidth}{0pt}
	\cleardoublepage
}



%----------------------------------------------------------------------------------------
%	TEXTO EN GENERAL
%----------------------------------------------------------------------------------------
\usepackage{parskip}							% Al usarlo, elimina los espacios entre parrafos
\setlength{\parindent}{0pt} 					% Elimina la sangría
\setlength{\parskip}{6pt plus 1pt minus 1pt} 	% Separacion base: 6pt, Maximo adicional 2pt y minimo adicional 1pt



%----------------------------------------------------------------------------------------
%	FORMATO PARA EL BLOQUE DE CODIGO EN LENGUAJE C
%----------------------------------------------------------------------------------------
\usepackage{listings}	% Permite bloques de codigo

\definecolor{mGreen}{rgb}{0,0.6,0}
\definecolor{mGray}{rgb}{0.5,0.5,0.5}
\definecolor{mPurple}{rgb}{0.58,0,0.82}
\definecolor{backgroundColour}{rgb}{0.95,0.95,0.92}

\lstdefinestyle{CStyle}{
	backgroundcolor=\color{backgroundColour},   
	commentstyle=\color{mGreen},
	keywordstyle=\color{magenta},
	numberstyle=\tiny\color{mGray},
	stringstyle=\color{mPurple},
	basicstyle=\ttfamily\footnotesize,
	breakatwhitespace=false,         
	captionpos=b,                    
	keepspaces=true,                 
	numbers=left,                    
	numbersep=1mm,                  
	showspaces=false,                
	showstringspaces=false,
	showtabs=false,                  
	tabsize=2,
	language=C,
	xleftmargin=3mm, 
	xrightmargin=2mm,
}

\definecolor{codegreen}{rgb}{0,0.6,0}
\definecolor{codegray}{rgb}{0.5,0.5,0.5}
\definecolor{codepurple}{rgb}{0.58,0,0.82}
\definecolor{codeblue}{rgb}{0.2, 0.58, 0.9}
\definecolor{backcolour}{rgb}{0.95,0.95,0.92}

\lstdefinestyle{Matlab}{
	language=Matlab,
	backgroundcolor=\color{backcolour},   
	commentstyle=\color{codegreen},
	keywordstyle=\color{codepurple},
	numberstyle=\tiny\color{codegray},
	stringstyle=\color{codeblue},
	basicstyle=\ttfamily\footnotesize,
	morekeywords={function,writeline,readline,str2double,flush,writeread,split},
	deletekeywords={input},
	morestring=*[d]{"},
	numbers=left,       
	numbersep=-5pt,        
	showstringspaces=false,
	tabsize=2,
	xleftmargin=0.6cm,
	xrightmargin=0.6cm,
	framexleftmargin=10pt,
	framexrightmargin=10pt,
}

\lstdefinestyle{rawStyle}{
	numberstyle=\tiny\color{mGray},
	basicstyle=\ttfamily,
	numbers=left,                    
	numbersep=5pt,                  
	xleftmargin=0.7cm
}




%----------------------------------------------------------------------------------------
%	CONFIGURACION DEL CUERPO (TABLAS Y FIGURAS)
%----------------------------------------------------------------------------------------
\usepackage{booktabs}		% Para centrar las tablas
%\newcommand{\configCuerpo}{
	%	\renewcommand{\figurename}{Figura}
	%	\renewcommand{\tablename}{Tabla}
	%}
\definecolor{GrayCaptions}{rgb}{0.5,0.5,0.5}
\usepackage[
font={color=GrayCaptions},
figurename=Imagen,
tablename=Tabla,
labelfont={it}
]{caption}
\setlength{\abovecaptionskip}{5pt plus 2pt minus 2pt} % Separacion de los 'captions' con el objeto
\setlength{\belowcaptionskip}{5pt plus 2pt minus 2pt} % Separacion de los 'captions' con el texto que continua.





%----------------------------------------------------------------------------------------
%	FORMATO DE LOS CAPITULOS
%----------------------------------------------------------------------------------------
\titleformat{\chapter}[display]
{\normalfont\huge\bfseries}							% Formato general
{\vspace{-2cm}\huge\chaptertitlename\ \thechapter}	% Formato y contenido del subtitulo
{20pt}												% Separacion entre ambos
{\Huge}												% Codigo a ejecutar antes del Subtitulo



%----------------------------------------------------------------------------------------
%	BOXS PERSONALIZADOS
%----------------------------------------------------------------------------------------
\newtcolorbox{boxDef}[1]{
	center,
	width=0.9\textwidth,
	arc=0.2mm,
	boxrule=1pt,
	colframe=black,
	left=3pt,   % Relleno a la izquierda
	right=3pt,  % Relleno a la derecha
	top=3pt,    % Relleno en la parte superior
	bottom=3pt, % Relleno en la parte inferior
	title=#1, 	% Utiliza el argumento para el título
}




%----------------------------------------------------------------------------------------
%	BIBLIOGRAFIA (Para deshabilitar, comentar lo siguiente)
%----------------------------------------------------------------------------------------
\usepackage{csquotes}
\usepackage[
	backend=biber,
	style=apa,
	sortcites,
	url=true
]{biblatex}
\addbibresource{Bibliografia.bib}

